\documentclass[size=a4, parskip=half, titlepage=false, toc=flat, toc=bib, 12pt]{scrartcl}

\setuptoc{toc}{leveldown}

% Ajuste de las líneas y párrafos
\linespread{1.2}
\setlength{\parindent}{0pt}
\setlength{\parskip}{12pt}

% Español
\usepackage[spanish, es-tabla]{babel}

% Matemáticas
\usepackage{amsmath}
\usepackage{amsthm}

% Links
%\usepackage{hyperref}


% Fuentes
\usepackage{newpxtext,newpxmath}
\usepackage[scale=.9]{FiraMono}
\usepackage{FiraSans}
\usepackage[T1]{fontenc}

\defaultfontfeatures{Ligatures=TeX,Numbers=Lining}
\usepackage[activate={true,nocompatibility},final,tracking=true,factor=1100,stretch=10,shrink=10]{microtype}
\SetTracking{encoding={*}, shape=sc}{0}

\usepackage{graphicx}
\usepackage{float}

% Mejores tablas
\usepackage{booktabs}

\usepackage{adjustbox}

% COLORES

\usepackage{xcolor}

\definecolor{verde}{HTML}{007D51}
\definecolor{esmeralda}{HTML}{045D56}
\definecolor{salmon}{HTML}{FF6859}
\definecolor{amarillo}{HTML}{FFAC12}
\definecolor{morado}{HTML}{A932FF}
\definecolor{azul}{HTML}{0082FB}
\definecolor{error}{HTML}{b00020}

% ENTORNOS
\usepackage[skins, listings, theorems]{tcolorbox}

\newtcolorbox{recuerda}{
  enhanced,
%  sharp corners,
  frame hidden,
  colback=black!10,
	lefttitle=0pt,
  coltitle=black,
  fonttitle=\bfseries\sffamily\scshape,
  titlerule=0.8mm,
  titlerule style=black,
  title=\raisebox{-0.6ex}{\small RECUERDA}
}

\newtcolorbox{nota}{
  enhanced,
%  sharp corners,
  frame hidden,
  colback=black!10,
	lefttitle=0pt,
  coltitle=black,
  fonttitle=\bfseries\sffamily\scshape,
  titlerule=0.8mm,
  titlerule style=black,
  title=\raisebox{-0.6ex}{\small NOTA}
}

\newtcolorbox{error}{
  enhanced,
%  sharp corners,
  frame hidden,
  colback=error!10,
	lefttitle=0pt,
  coltitle=error,
  fonttitle=\bfseries\sffamily\scshape,
  titlerule=0.8mm,
  titlerule style=error,
  title=\raisebox{-0.6ex}{\small ERROR}
}

\newtcblisting{shell}{
  enhanced,
  colback=black!10,
  colupper=black,
  frame hidden,
  opacityback=0,
  coltitle=black,
  fonttitle=\bfseries\sffamily\scshape,
  %titlerule=0.8mm,
  %titlerule style=black,
  %title=Consola,
  listing only,
  listing options={
    style=tcblatex,
    language=sh,
    breaklines=true,
    postbreak=\mbox{\textcolor{black}{$\hookrightarrow$}\space},
    emph={jmml@UbuntuServer, jmml@CentOS},
    emphstyle={\bfseries},
  },
}

\newtcbtheorem[number within=section]{teor}{\small TEOREMA}{
  enhanced,
  sharp corners,
  frame hidden,
  colback=white,
  coltitle=black,
  fonttitle=\bfseries\sffamily,
  %separator sign=\raisebox{-0.65ex}{\Large\MI\symbol{58828}},
  description font=\itshape
}{teor}

\newtcbtheorem[number within=section]{prop}{\small PROPOSICIÓN}{
  enhanced,
  sharp corners,
  frame hidden,
  colback=white,
  coltitle=black,
  fonttitle=\bfseries\sffamily,
  %separator sign=\raisebox{-0.65ex}{\Large\MI\symbol{58828}},
  description font=\itshape
}{prop}

\newtcbtheorem[number within=section]{cor}{\small COROLARIO}{
  enhanced,
  sharp corners,
  frame hidden,
  colback=white,
  coltitle=black,
  fonttitle=\bfseries\sffamily,
  %separator sign=\raisebox{-0.65ex}{\Large\MI\symbol{58828}},
  description font=\itshape
}{cor}

\newtcbtheorem[number within=section]{defi}{\small DEFINICIÓN}{
  enhanced,
  sharp corners,
  frame hidden,
  colback=white,
  coltitle=black,
  fonttitle=\bfseries\sffamily,
  %separator sign=\raisebox{-0.65ex}{\Large\MI\symbol{58828}},
  description font=\itshape
}{defi}

\newtcbtheorem{ejer}{\small EJERCICIO}{
  enhanced,
  sharp corners,
  frame hidden,
  left=0mm,
  right=0mm,
  colback=white,
  coltitle=black,
  fonttitle=\bfseries\sffamily,
  %separator sign=\raisebox{-0.65ex}{\Large\MI\symbol{58828}},
  description font=\itshape,
  nameref/.style={},
}{ejer}


% CÓDIGO
\usepackage{listings}

% CABECERAS
\pagestyle{headings}
\setkomafont{pageheadfoot}{\normalfont\normalcolor\sffamily\small}
\setkomafont{pagenumber}{\normalfont\sffamily}

% ALGORITMOS
\usepackage[vlined,linesnumbered]{algorithm2e}

% Formato de los pies de figura
\setkomafont{captionlabel}{\scshape}
\SetAlCapFnt{\normalfont\scshape}
\SetAlgorithmName{Algoritmo}{Algoritmo}{Lista de algoritmos}

% BIBLIOGRAFÍA
%\usepackage[sorting=none]{biblatex}
%\addbibresource{bibliografia.bib}


\begin{document}

\renewcommand{\proofname}{\normalfont\sffamily\bfseries\small DEMOSTRACIÓN}

\title{Proyecto final\\
ShapeContext}
\subject{Visión por computador}
\author{Johanna Capote Robayna\\
    Guillermo Galindo Ortuño\\
    5 del Doble Grado en Informática y Matemáticas\\
    Grupo A}
\date{}
\publishers{\vspace{2cm}\includegraphics[height=2.5cm]{UGR}\vspace{1cm}}
\maketitle

\newpage

\tableofcontents
\newpage

\section{Introducción}
En este proyecto vamos a implementar el algoritmo de \textit{ShapeContext}, utilizado para resolver el problema coincidencia de imágenes. Empleado para el reconocimiento de texto y dígitos escritos a mano o huellas dactilares.

La principal idea del este algoritmo se basa construir un histograma basado en puntos en el sistema de coordenadas porales. Para cada punto, se obtiene la distancia y los ángulos euclidianos con respecto a otros puntos, se normalizan y se representa en las regiones del mapa el número de puntos de cada registro.

\begin{center}
\includegraphics[height=6cm]{./img/intro}
\end{center}

\newpage

\section{Implementación}
Para implementar este algoritmo, se ha declarado una clase \verb|ShapeContext|. En el constructor declaramos los parámetros sobre los que construimos las regiones en las que se divide la imagen. Es decir, el radio del circulo interior (\verb|r_inner|), el radio del último circulo exterior (\verb|r_outer|), el número de círculos concentricos (\verb|nbins_r|) y el número de regiones en las que se divide los círculos (\verb|nbins_theta|).

\begin{verbatim}
    self.nbins_r = 5
    # number of angles zones
    self.nbins_theta = 12
    # maximum and minimum radius
    self.r_inner = 0.1250
    self.r_outer = 2.0
\end{verbatim}

El esqueleto del algoritmo se encuentra en la funcion \verb|compute(self, points)|.
\begin{enumerate}
\item En primer lugar calculamos la distancia entre los puntos y los normalizamos por la media, además obtenemos los dos puntos con distancia máxima.

\begin{verbatim}
    t_points = len(points)
    # getting euclidian distance
    r_array = cdist(points, points)
    # getting two points with maximum distance to norm angle by them
    # this is needed for rotation invariant feature
    am = r_array.argmax()
    max_points = [am / t_points, am % t_points]
    # normalizing
    r_array_n = r_array / r_array.mean()
\end{verbatim}

\item En segundo lugar creamos los registros y contamos el número de puntos que se encuentran encerrados en cada contenedor. Para ello creamos una matriz de 0 cuadrada de longitud el número de puntos, que usaremos como contador para saber en que regiones se encuentra. En cada iteración comprobamos que los puntos sean menores que el el radio del círculo de la región que estamos testeando, si es menor, aumentamos el contador en uno.

Por ejemplo si nos encontramos en un espacio logaritmo con los siguientes intervalos.
logspace $= [\decimalpoint 0.1250, 0.2500, 0.5000, 1.0000, 2.0000]$ y tenemos la siguiente matriz de puntos:
$$\decimalpoint \begin{bmatrix}
0 & 1.3 \\
0.43 & 0
\end{bmatrix}$$
La matriz \verb|r_array_q| iría aumentando de la siguiente forma:
$$\decimalpoint \begin{bmatrix}
0 & 0 \\
0 & 0
\end{bmatrix} \stackrel{0 < 0.125}{\longrightarrow} \begin{bmatrix}
1 & 0 \\
0 & 1
\end{bmatrix} \stackrel{0 < 0.25}{\longrightarrow} \begin{bmatrix}
2 & 0 \\
0 & 2
\end{bmatrix} \stackrel{0 < 0.5}{\longrightarrow} \begin{bmatrix}
3 & 0 \\
1 & 3
\end{bmatrix} \stackrel{0.43 < 1.0}{\longrightarrow} \begin{bmatrix}
4 & 0 \\
2 & 4
\end{bmatrix} \stackrel{1.3 < 2.0}\rightarrow \begin{bmatrix}
5 & 1 \\
3 & 5
\end{bmatrix}$$

\begin{verbatim}
r_bin_edges = np.logspace(np.log10(r_inner), np.log10(r_outer), nbins_r)
r_array_q = np.zeros((t_points, t_points), dtype=int)
for m in xrange(self.nbins_r):
    r_array_q += (r_array_n < r_bin_edges[m])
\end{verbatim}
\end{enumerate}


%printbibliography


\end{document}
